\documentclass{article}
\usepackage{graphicx} % Required for inserting images
\usepackage{amsmath, amssymb, amsthm} % For mathematical symbols and theorem environments
\usepackage{mathrsfs} % For the \mathscr command for the algebra R
\newtheorem{conjecture}{Conjecture} % Defines a numbered Conjecture environment
\newtheorem{definition}{Definition}[section]
\newtheorem{axiom}{Axiom}[section]
\theoremstyle{remark}
\newtheorem{remark}{Remark}
\newtheorem{theorem}[definition]{Theorem}
\usepackage[english]{babel}
\usepackage[autostyle]{csquotes} 
\usepackage{amssymb}
\newtheorem{lemma}{Lemma}
\newtheorem{example}{Example}

\begin{document}
% Custom commands for Reciprocal Number elements
\newcommand{\rnAlgebra}{\mathscr{R}}
\newcommand{\eps}{\varepsilon}
\newcommand{\om}{\omega}
\newcommand{\Om}{\Omega}
\newcommand{\epE}{\varepsilon E}
\newcommand{\Eep}{E\varepsilon}
\newcommand{\omE}{\omega E}
\newcommand{\Eom}{E\omega}
\newcommand{\epOm}{\varepsilon \Omega}
\newcommand{\Omep}{\Omega \varepsilon}

\title{Reciprocal Numbers: My Conjurings}
\author{Ryan Wang}
\date{Made Sometime September + October Edits 2025, Thought about ~December 2024, Revisted ~June 2025}


\maketitle

\section{Algebraic Structure Definition}

\subsection{Basic Algebraic Framework}

\begin{definition}[Reciprocal Number Algebra(Magma])
The Reciprocal Number algebra(Magma) $\rnAlgebra$ is a Non-Associative, Non-Distributive algebra over the real numbers where the real numbers are embedded into $\rnAlgebra$ as a real scalars commute and associate with all generators. In particular $ r \cdot \eps = \eps \cdot r$ for every $r \in \mathbb{R}$ generated by the symbols:
\[
\{\varepsilon, E, \omega, \Omega, 0, \infty\}
\]
with the defining relations given below.
\end{definition}

\subsection{Fundamental Relations}
\begin{axiom}[Scalar Linearity within Pure Scales]
Real scalars act bilinearly and commute with elements inside any pure-scale subalgebra. Scalar bilinearity across mixed-scale expressions is permitted only when the distinct irreducible element count $\le 2$. Bilinearity extends to reciprocal pairs $(\eps, \om)$ and $(E, \Om)$, which are treated as conjugate scales forming a closed subalgebra.
\end{axiom}

\textbf{Nilpotency Relations:}
\begin{align*}
    \varepsilon^2 &= 0, \quad \varepsilon^n = 0 \text{ for } n > 1
\end{align*}

\textbf{Reciprocal Relations:}
\begin{align*}
    \varepsilon \cdot \omega = \omega \cdot \varepsilon = 1, \quad E \cdot \Omega = \Omega \cdot E = 1
\end{align*}

\begin{theorem}[Failure of general associativity]
The algebra $\rnAlgebra$ is not associative: there exist $a,b,c\in\rnAlgebra$ for which $a(bc)\neq (ab)c$.
\end{theorem}

\begin{proof}
Assume, for contradiction, that multiplication in $\rnAlgebra$ is associative for all elements. Consider the triple $(\varepsilon,\Omega,\infty)$. Compute the two parenthesizations using the axioms and the multiplication table.

Left-association:
\[
\varepsilon\big(\Omega\infty\big)=\varepsilon(\Omega\infty).
\]
By the table $\Omega\infty$ is a named element and, using $\varepsilon\cdot\infty=\omega$, we obtain
\[
\varepsilon(\Omega\infty)=\varepsilon\infty=\omega.
\]

Right-association:
\[
(\varepsilon\Omega)\infty.
\]
By the table $\varepsilon\Omega$ is a distinct mixed reciprocal symbol; denote $X:=\varepsilon\Omega$. Then $(\varepsilon\Omega)\infty = X\infty$, which is a symbol distinct from $\omega$ by the scale linear-independence axioms. Hence
\[
\varepsilon(\Omega\infty)=\omega \neq X\infty=(\varepsilon\Omega)\infty,
\]
contradicting associativity. Therefore multiplication is not associative. 
\end{proof}

\begin{axiom}[Restricted Associativity]
For all elements $a,b,c \in \rnAlgebra$, associativity of multiplication holds if the product involves at most two distinct irreducible elements, where:

\begin{itemize}
\item Each occurrence of $0$ or $\infty$ counts as one distinct irreducible element (regardless of scalar coefficients)
\item Real scalars do not count toward the irreducible element count  
\item Composite irreducible elements (e.g., $a = x + \varepsilon$ where $x \in \mathbb{R}$) count as one irreducible element
\item Scalar multiplication does not necessarily commute with terminal elements $0$ and $\infty$
\end{itemize}

Formally:
\[
(a b) c = a (b c)
\quad \text{if} \quad \text{count of distinct irreducible elements in } \{a,b,c\} \le 2
\]
where each occurrence of $0$ and $\infty$ is counted separately.
\end{axiom}

\begin{axiom}[Scalar Commutation with Non-Terminal Elements]
For $r \in \mathbb{R}$ and $a \in \{\varepsilon, E, \omega, \Omega\}$:
\[
r \cdot a = a \cdot r
\]
Real scalars commute with non-terminal generators.
\end{axiom}

\begin{axiom}[Scalar Non-Commutation with Terminal Elements]
For $r \in \mathbb{R}$ and terminal elements:
\[
r \cdot 0 = 0 \cdot r \quad \text{and} \quad r \cdot \infty = \infty \cdot r
\]
do not necessarily hold unless $r = 1$. \\
Basically: Is it attached to the terminal or expression?
\end{axiom}

\begin{definition}[Irreducible Element]
An element $a \in \rnAlgebra$ is \emph{irreducible} if it cannot be further simplified. Examples:
\begin{itemize}
\item Pure scale elements: $\varepsilon, E, \omega, \Omega, 0, \infty$
\item Mixed scale composites: $\epE, \Eep, \epOm, \Omep$
\item Terminal composites: $0, \infty$ (each counts as one irreducible)
\item Exponentiated elements: $E^2, \Omega^2, \Om^5$
\item Additive composites: $(r+\eps),(\eps+\om),(E+\eps) $
\end{itemize}
\end{definition}

\begin{definition}[Distinct Irreducible Count]
For elements $a_1, a_2, \ldots, a_n \in \rnAlgebra$, the \emph{distinct irreducible count} is determined by:
\begin{enumerate}
\item Count each distinct irreducible type, ignoring real scalar coefficients
\item Count each occurrence of $0$ and $\infty$ separately, regardless of scalar coefficients
\item Ignore the count of real scalars themselves
\end{enumerate}
\end{definition}

\begin{theorem}[Failure of general associativity - Revised]
The algebra $\rnAlgebra$ is not associative: there exist $a,b,c\in\rnAlgebra$ for which $a(bc)\neq (ab)c$.
\end{theorem}

\begin{proof}
Consider the triple $(\varepsilon,\Omega,\infty)$. We count distinct irreducible elements:
\begin{itemize}
\item $\varepsilon$: 1 irreducible element
\item $\Omega$: 1 irreducible element  
\item $\infty$: 1 irreducible element (counted separately)
\end{itemize}
Total: 3 distinct irreducible elements, so associativity is not guaranteed.

Now compute both parenthesizations:

Left-association:
\[
\varepsilon(\Omega\infty) = \varepsilon(\Omega\infty)
\]
By the multiplication table, $\Omega\infty$ is a named element, and using $\varepsilon\cdot\infty=\omega$:
\[
\varepsilon(\Omega\infty) = \varepsilon\infty = \omega
\]

Right-association:
\[
(\varepsilon\Omega)\infty = (\varepsilon\Omega)\infty
\]
By the multiplication table, $\varepsilon\Omega$ is the distinct mixed reciprocal symbol $\epOm$, so:
\[
(\varepsilon\Omega)\infty = \epOm\infty
\]
which is a symbol distinct from $\omega$ by scale linear-independence axioms.

Therefore:
\[
\varepsilon(\Omega\infty) = \omega \neq \epOm\infty = (\varepsilon\Omega)\infty
\]
contradicting associativity.
\end{proof}

\subsubsection{Examples of Associativity Testing}

\begin{example}[Associative Case: 2 Irreducible Elements]
Consider $(\varepsilon, \omega, \varepsilon)$:
\begin{itemize}
\item Irreducible elements: $\varepsilon, \omega$ (2 distinct types)
\item $(\varepsilon\omega)\varepsilon = 1\cdot\varepsilon = \varepsilon$
\item $\varepsilon(\omega\varepsilon) = \varepsilon\cdot 1 = \varepsilon$
\item Associativity holds: $\checkmark$
\end{itemize}
\end{example}

\begin{example}[Non-Associative Case: 3 Irreducible Elements]  
Consider $(\varepsilon, \Omega, \infty)$:
\begin{itemize}
\item Irreducible elements: $\varepsilon, \Omega, \infty$ (3 distinct types)
\item $(\varepsilon\Omega)\infty = (\epOm)\infty$ (new irreducible)
\item $\varepsilon(\Omega\infty) = \varepsilon\infty = \omega$ (different irreducible)
\item Associativity fails: $\times$
\end{itemize}
\end{example}

\begin{example}[Terminal Element Counting]
Consider $(0, \varepsilon, \infty)$:
\begin{itemize}
\item Irreducible elements: $0, \varepsilon, \infty$ (3 distinct types, terminals counted separately)
\item $(0\cdot\varepsilon)\infty = 0\cdot\infty = 1$
\item $0\cdot(\varepsilon\infty) = 0\cdot\omega = \varepsilon$  
\item Associativity fails: $\times$
\end{itemize}
\end{example}

\begin{theorem}[Failure of distributivity]
The algebra $\rnAlgebra$ is not distributive: there exist $a,b,c\in\rnAlgebra$ for which $a(b+c)\neq ab+ac$.
\end{theorem}

\begin{proof}
Assume distributivity holds universally. Take $a=0$ and $b=c=\infty$. Then distributivity implies
\[
0(\infty+\infty)=0\infty + 0\infty.
\]
By the Infinity--Zero axiom, $0\infty=1$. Hence the right-hand side equals $1+1$. But the left-hand side evaluates to $0(\infty+\infty)=0\infty=1$ (by the terminal addition and simplification rules for $\infty$). Thus we obtain
\[
1 = 0(\infty+\infty) = 0\infty = 1 \quad\text{but}\quad 0\infty + 0\infty = 1+1 \neq 1,
\]
a contradiction. Therefore distributivity fails on mixed scales.
\end{proof}

\textbf{Infinity-Zero Relations:}
\begin{align*}
    0 \cdot \infty = \infty \cdot 0 = 1
\end{align*}

\textbf{Extended Relations (just in case we forgot):}
\begin{align*}
    \omega^2 &= \infty \\
    \omega \cdot 0 &= \varepsilon \\
    \varepsilon \cdot \infty &= \omega
\end{align*}

\section{Complete Multiplication Table}

\subsection{Base Multiplication Table}

The following table defines the product $R \times C$:
\[
\begin{array}{c|ccccccc}
\times & 1 & \varepsilon & E & \omega & \Omega & 0 & \infty \\
\hline
1      & 1 & \varepsilon & E & \omega & \Omega & 0 & \infty \\
\varepsilon & \varepsilon & 0 & \epE & 1 & \epOm & 0 & \omega \\
E       & E & \Eep & E^2 & \Eom & 1 & 0 & E\infty \\
\omega  & \omega & 1 & \omE & \infty & \omega\Omega & \varepsilon & \infty \\
\Omega & \Omega & \Omep & 1 & \Omega\omega & \Omega^2 & 0\Omega & \Omega\infty \\
0       & 0 & 0 & 0 & \varepsilon & 0\Omega & 0 & 1 \\
\infty & \infty & \omega & \infty E & \infty & \infty & 1 & \infty \\
\end{array}
\]
\begin{axiom}[Irreducible Basis and Table Authority]
The multiplication table defines irreducible basis symbols. Any product not reduced by the table is a new irreducible basis symbol. These symbols are mutually distinct unless an explicit simplification is stated.
\end{axiom}
\subsection{Derived Elements and Their Products}

\textbf{Composite Elements:}
\begin{itemize}
    \item $\epE, \Eep$: Mixed scale terms
    \item $E^2, \Omega^2$: Higher-order countable terms
    \item $\omE, \Eom$: Mixed infinite terms
    \item $\epOm, \Omep$: Mixed reciprocal terms
\end{itemize}

\textbf{Key Properties:}
\begin{align*}
    (\epE)^2 = 0, \quad (\Eep)^2 = 0 \quad &\text{(nilpotent)} \\
    E^2 \neq 0, \quad \Omega^2 \neq \infty \quad &\text{(non-nilpotent and non-anti-nilpotent)}
\end{align*}

\section{Mathematical Structure Properties}

\subsection{Algebraic Structure}

\begin{definition}[Scale Hierarchy]
The elements have a natural scale ordering:
\[
0 \ll \varepsilon \ll E \ll 1 \ll \Omega \ll \omega \ll \infty
\]
where $\ll$ denotes ``much smaller in scale''.
\end{definition}

\subsection{Special Elements and Their Properties}

\textbf{Zero Element (0):}
\begin{itemize}
    \item $0 \cdot a = 0$ for $a \ll \Omega$
    \item $0 \cdot \infty = 1$
    \item $0 + a = a$ for all $a$
\end{itemize}

\textbf{Infinity Element ($\infty$):}
\begin{itemize}
    \item $\infty \cdot a = \infty$ for $a \ll E$
    \item $\infty \cdot 0 = 1$
    \item $\infty + a = \infty$ for all a
\end{itemize}

\subsection{Extended Number Representation}

\begin{definition}[General RN Number]
Any element of $\rnAlgebra$ can be written as:
\[
a = a_0 + a_\varepsilon \varepsilon + a_E E + a_{\epE} \epE + a_{E^2} E^2 + \cdots
\]
where $a_0, a_\varepsilon, a_E, \ldots \in \mathbb{R}$.

\end{definition}
\section{Addition in Reciprocal Numbers}

\subsection{Scale Independence Principle}

\begin{axiom}[Scale Linear Independence]
All scale elements are linearly independent under addition. No scale can be expressed as a linear combination of other scales, and addition preserves scale separation unless explicit simplification rules apply.
\end{axiom}

\begin{definition}[Formal Addition]
For any elements $a, b \in \rnAlgebra$, their sum $a + b$ is treated as a formal sum unless specific simplification rules apply. The algebra maintains scale separation such that:
\[
a\varepsilon + bE \neq (a + b)X \quad \text{for any single scale } X
\]
\end{definition}

\subsection{Simplification Rules}

\begin{definition}[Terminal Addition]
\begin{align*}
    % Zero is additive identity
    0 + a &= a + 0 = a \quad \text{for all } a \in \rnAlgebra \\
    % Infinity absorbs everything
    \infty + a &= a + \infty = \infty \quad \text{for all } a \in \rnAlgebra \\
    % Same-scale addition works normally
    a\varepsilon + b\varepsilon &= (a + b)\varepsilon \quad \text{for } a, b \in \mathbb{R} \\
    aE + bE &= (a + b)E \quad \text{for } a, b \in \mathbb{R} \\
    a\omega + b\omega &= (a + b)\omega \quad \text{for } a, b \in \mathbb{R} \\
    a\Omega + b\Omega &= (a + b)\Omega \quad \text{for } a, b \in \mathbb{R}
\end{align*}
\end{definition}
Note: Distributivity does not hold for mixed scale


\subsection{Extended Number Addition}

\begin{definition}[Addition of General RN Numbers]
For general RN numbers in expanded form:
\begin{align*}
a &= a_0 + a_\varepsilon \varepsilon + a_E E + a_\omega \omega + a_\Omega \Omega + \cdots \\
b &= b_0 + b_\varepsilon \varepsilon + b_E E + b_\omega \omega + b_\Omega \Omega + \cdots
\end{align*}
Their sum is the formal sum:
\[
a + b = (a_0 + b_0) + (a_\varepsilon + b_\varepsilon)\varepsilon + (a_E + b_E)E + (a_\omega + b_\omega)\omega + (a_\Omega + b_\Omega)\Omega + \cdots
\]
with the understanding that components remain separated by scale unless terminal or same-scale rules apply.
\end{definition}

\subsection{Properties of Addition}

\begin{theorem}[Limited Algebraic Structure]
\begin{itemize}
    \item \textbf{Commutative:} $a + b = b + a$ for all $a, b \in \rnAlgebra$
    \item \textbf{Associative:} $(a + b) + c = a + (b + c)$ for all $a, b, c \in \rnAlgebra$
    \item \textbf{Zero Identity:} $0 + a = a + 0 = a$ for all $a \in \rnAlgebra$
    \item \textbf{Infinite Absorption:} $\infty + a = \infty$ for all $a \in \rnAlgebra$
\end{itemize}
\end{theorem}

\subsection{Comparison with Dual Numbers}

\begin{remark}[Dual Number Parallel]
In dual numbers, expressions like $\varepsilon + 4$ are left unsimplified because $\varepsilon$ and $1$ exist at different scales. Similarly, in Reciprocal Numbers, we maintain:
\[
\varepsilon + E \quad \text{remains as } \varepsilon + E
\]
because these represent fundamentally different cardinalities and cannot be combined.
\end{remark}

\begin{remark}[Norm Incompatibility]
The scale separation makes traditional norms impossible because elements are not orthogonal in the vector space sense, but rather exist in fundamentally different dimensionalities. A "scale-valued norm" would be required to properly characterize distances in this algebra.
\end{remark}
\section{Fundamental Mathematical Properties}

\subsection{Key Identities}

\textbf{Nilpotency Identities:}
\[
\varepsilon^2 = 0, \quad (\epE)^2 = 0, \quad (\Eep)^2 = 0
\]

\textbf{Reciprocal Identities:}
\[
\varepsilon \omega = \omega \varepsilon = 1, \quad E \Omega = \Omega E = 1
, \quad 0\infty=\infty0=1\]
\subsection{RN-Valued Functions}

\begin{definition}[RN Function Space]
Let $V_\rnAlgebra$ be the space of functions $f: \mathbb{R}^n \to \rnAlgebra$ with expansion:
\[
f(x) = f_0(x) + f_\varepsilon(x) \varepsilon + f_E(x) E + f_{\epE}(x) \epE + \cdots
\]
\end{definition}

\section{Calculus Operations}

\subsection{Algebraic Differentiation}

\begin{definition}[RN Derivative]
For $f \in V_\rnAlgebra$, define:
\[
\partial_x f := \text{coefficient of } \varepsilon \text{ in } f(x + \varepsilon)
\]
This agrees with the standard derivative and is possible because in the $\eps$ subalgebra associativity and distributivity holds, this method is a direct implementation of Dual Number Differentiation within the RN $\epsilon$-subalgebra. TLDR: Dual Derivative
\end{definition}

\begin{conjecture}[RN Integration]
There exists some method of using $\omega$ to achieve automatic integration
\end{conjecture}

\section{Mathematical Structure Summary}

\subsection{Neat Stuff}

\begin{enumerate}
    \item \textbf{Scale-Dependent Algebra:} Operations behave differently at different scales.
    \item \textbf{Scale-Separated Dimensions:} Elements are scale-separated: they are too far apart in magnitude
    \item \textbf{Built-in Calculus:} Derivatives emerge as algebraic operations(Maybe int soon).
    \item \textbf{Cardinality Awareness:} Distinction between countable and uncountable scales.
    \item \textbf{Non-Standard Arithmetic:} Modified rules for zero and infinity.
    \item \textbf{Structural Regularization:} The inherent structure is conjectured to provide an algebraic regularization mechanism for handling the discontinuous solutions found in non-linear partial differential equations.
\end{enumerate}

\subsection{Fundamental Structure}

\begin{conjecture}[Scale Separation]
The algebra naturally separates into scales:
\begin{itemize}
    \item $\varepsilon$-scale: Continuous, nilpotent operations
    \item $E$-scale: Discrete, non-nilpotent operations
    \item Mixed-scale: Transitional regions
\end{itemize}
\end{conjecture}

\section{Comparison with Other Number Systems}
Reciprocal Numbers sits is like several classical extensions of the reals, while remaining distinct in key ways. It is like an expansion of the \emph{dual numbers} (nilpotent $\varepsilon$ with $\varepsilon^2=0$ and are associative, useful for automatic differentiation), and it shares with \emph{nonstandard analysis}/the \emph{hyperreal} field the explicit presence of infinitesimals and infinities.  It also parallels constructions to the surreal numbers, which provide a very large ordered field containing both infinities and infinitesimals.

Two features distinguish Reciprocal Numbers from those systems: (i) \textbf{tiered reciprocal pairs} (countable vs. uncountable vs. terminal) are primitive and built into the multiplication table, and (ii) the algebra is intentionally \emph{non-associative} and \emph{non-distributive} on mixed scales to save against the horrors of infinities, whereas dual numbers, hyperreals and surreals are associative (and most remain distributive). For an example of non-associative algebra, compare with the octonions: octonions sacrifice associativity while preserving weaker identities; Reciprocal Numbers likewise sacrifices full associativity to preserve scale-dependent reciprocity, special zero–infinity interactions, and weird associativity rule.
\section{Worked Example: Algebraic Regularization of a Burgers-Type Singularity}

Lets see how the Reciprocal Number (RN) algebra could regularizes an infinite derivative into a finite discontinuity, using a Burgers-type equation as motivation.

\subsection*{RN Axioms Used}
\begin{align}
    \varepsilon^2 &= 0, \label{eq:epsilon2}\\
    \omega\varepsilon &= 1, \label{eq:reciprocal}\\
    r\varepsilon &= \varepsilon r, \quad r\omega = \omega r \quad \forall r\in\mathbb{R}.
\end{align}
Bilinearity, associativity, and distributivity are valid inside pure subalgebras 
(i.e.\ among reals and powers of $\varepsilon$, or among reals and powers of $\omega$), 
but are not assumed to hold across mixed-scale terms except as explicitly declared 
in \eqref{eq:reciprocal}.  

\subsection*{Step 1. RN Derivative}

For a scalar field $u(x)$, the RN expansion at a point $x$ is
\begin{equation}
    u(x+\varepsilon) = u(x) + u_{\varepsilon}\,\varepsilon,
\end{equation}
where $u_{\varepsilon}$ is the RN coefficient corresponding to $\partial_x u$.
If $u_{\varepsilon}\in\mathbb{R}$, the function is locally smooth.  

At a singular point where the gradient blows up, we represent this as
\begin{equation}
    u_{\varepsilon} = k\,\omega, \quad k\in\mathbb{R}.
\end{equation}
We use $\omega$ because the derivative is not a divide by zero but rather a divide by $\eps$ error.
\subsection*{Step 2. Regularization of the Infinite Derivative}

Substituting into the RN expansion,
\begin{align}
    u(x+\varepsilon) &= u(x) + (k\,\omega)\varepsilon \\
                     &= u(x) + k(\omega\varepsilon) \\
                     &= u(x) + k, \quad \text{by \eqref{eq:reciprocal}.}
\end{align}
Thus, the infinite slope $k\omega$ over an infinitesimal distance $\varepsilon$
produces a finite jump of size $k$.  
Define
\begin{equation}
    u_L := u(x), \qquad u_R := u(x+\varepsilon) = u_L + k.
\end{equation}
This represents an instantaneous shock of amplitude $k$.

\subsection*{Step 3. Shock Speed for Inviscid Burgers Equation}

For the inviscid Burgers equation
\begin{equation}
    u_t + \tfrac{1}{2}\partial_x(u^2) = 0,
\end{equation}
the Rankine--Hugoniot jump condition gives the shock speed
\begin{align}
    s &= \frac{f(u_R)-f(u_L)}{u_R-u_L}, \quad f(u)=\tfrac{1}{2}u^2,\\
      &= \frac{\tfrac{1}{2}(u_L+k)^2 - \tfrac{1}{2}u_L^2}{k}
       = u_L + \tfrac{k}{2}.
\end{align}

\subsection*{Step 4. Numerical Example}

Take $u_L = 2$ and $k = 3$. Then
\[
u_R = u_L + k = 5, \qquad
s = u_L + \tfrac{k}{2} = 3.5.
\]
Hence, the RN algebra converts an infinite slope into a finite discontinuity 
of size $3$, propagating at speed $3.5$.

\subsection*{Step 5. Remarks}

The derivation above uses only the relations \eqref{eq:epsilon2}--\eqref{eq:reciprocal}
and linearity in real coefficients, without applying distributivity across mixed scales.
The RN algebra therefore provides a consistent local regularization scheme:
\[
(\text{infinite slope})\times(\text{infinitesimal width}) = \text{finite jump}.
\]
This mechanism reproduces the structure of a Rankine--Hugoniot shock 
entirely through algebraic relations, with no limiting process required.
\section{Zundamon's Theorems/Acknowledgement}
Thanks to online resources (Zundamon’s Theorems and others) for introducing concepts of dual numbers and algebraic differentiation; without it I probably wouldn't have come up with RN(Reciprocal[ya]Numbers) or RyaN
\end{document}